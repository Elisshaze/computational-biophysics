\chapter{Introduction to molecular dynamics}

\section{Introduction}

	\subsection{Hamilton's equations}

	$$\dot{q}_\alpha = \frac{\partial\mathcal{H}}{\partial p_\alpha}\qquad\dot{p}_\alpha = - \frac{\partial\mathcal{H}}{\partial q_\alpha}$$

	$$\frac{d\mathcal{H}}{dt} = \sum\limits_\alpha\biggl[\frac{\partial\mathcal{H}}{\partial q_\alpha}\dot{q}_\alpha + \frac{\partial\mathcal{H}}{\partial p_\alpha}\dot{p}_\alpha\biggr] = \sum\limits_\alpha\biggl[\frac{\partial\mathcal{H}}{\partial q_\alpha}\frac{\partial\mathcal{H}}{\partial p_\alpha}-\frac{\partial\mathcal{H}}{\partial p_\alpha}\frac{\partial\mathcal{H}}{\partial q_\alpha}\biggr] = 0$$

	$$\mathcal{H}(q_1, \dots, q_{3N}, p_1, \dots, p_{3N}) = const$$

	\subsection{Ergodicity}

	$$A = \langle a\rangle = \frac{\int dxa(x)\delta(\mathcal{H}(x)-E)}{\int dx\delta(\mathcal{H}(x)-E)} = \lim\limits_{\tau\rightarrow\infty}\frac{1}{\tau}\int_0^\tau dta(x_t)\equiv\bar{a}$$

	The discretized time average:

	$$A = \langle a\rangle = \frac{1}{M}\sum\limits_{n=1}^M a(x_{n\Delta t})$$

	\subsection{Basic components of a molecular dynamics simulation}
	The basic components of a MD simulation are:

	\begin{multicols}{2}
		\begin{itemize}
			\item The model: the chosen force field.
			\item Calculation of energies and forces: accurate and efficient.
			\item The algorithm: used to integrate the equations of motion.
		\end{itemize}
	\end{multicols}

\section{Verlet algorithm}
Consider the Taylor expansion:

\begin{align*}
	\vec{r}_i(t+\Delta t)&\approx \vec{r}_i(t) + \Delta t\dot{\vec{r}}_i(t) + frac{1}{2}\Delta t^2\ddot{\vec{r}}_i(t)
											 &\approx\vec{r}_i(t+\Delta t)+\Delta t\vec{v}_i(t)+\frac{\Delta t^2}{2m_i}\vec{F}_i(t) \\
\end{align*}

Similarly:

$$\vec{r}_i(t-\Delta t) \approx\vec{r}_i(t)-\Delta t\vec{v}_i(t)+\frac{\Delta t^2}{2m_i}\vec{F}_i(t)$$

Now:

$$\vec{r}_i(t+\Delta t) + \vec{r}_i(t-\Delta t) = 2\vec{r}_i(t) + \frac{\Delta t^2}{m_i}\vec{f}_i(t)$$

Hence:

$$\vec{r}_i(t+\Delta t) = 2\vec{r}_i(t) - \vec{r}_i(t-\Delta t) + \frac{\Delta t^2}{m_i}\vec{F}_i(t)$$

And the velocities:

$$\vec{v}_i(t) = \frac{\vec{r}_i(t+\Delta t)-\vec{r}_i(t-\Delta t)}{2\Delta t}$$


	\subsection{Velocity Verlet}

	$$\vec{r}_i(t+\Delta t) = \vec{r}_i(t) + \Delta t\vec{v}_i(t) + \frac{\Delta t^2}{2m_i}\vec{F}_i(t)$$

	$$\vec{r}_i(t) = \vec{r}_i*t+\Delta t) -\Delta t\vec{v}_i(t + \Delta t) + \frac{\Delta t^2}{2m_i}\vec{F}_i(t+\Delta t)$$

	By sobstitution:

	$$\vec{v}_i(t+\Delta t) = \vec{v}_i(t) + \frac{\Delta t}{2m_i}[\vec{F}_i(t) + \vec{F}_i(t+\Delta t)]$$

	The time reversibility property comes from Hamilton's equation, while the symplectic structure gives numerical stability.

	\subsection{Initial condition}
	The coordinates of the initial condition are either taken from experimental data or guessed.
	The velocities are taken randomly from a Maxwell-Boltzmann distribution:

	$$f(v) = \biggl(\frac{m}{2\pi kT}\biggr)^\frac{1}{2}e^{-\frac{mv^2}{2kT}}$$

	Consider the Gaussian probability distribution:

	$$f(x) = \frac{1}{\sqrt{2\pi\sigma^2}}e^{-\frac{x^2}{2\sigma^2}}$$

	\subsection{Action integral}
	Consider:

	$$Q \equiv\{q_1, \dots, q_{3N}\}\qquad \dot{Q}\equiv\{\dot{q}_1, \dots, \dot{q}_{3N}\}$$

	The action integral:

	$$A[Q] = \int_{t_1}^{t_2}\mathcal{L}(Q(t), \dot{Q}(t))dt$$

	The path $Q$ that renders the action stationary is:

	\begin{multicols}{4}
		\begin{itemize}
			\item $Q(t_1) = Q_1$.
			\item $Q(t_2) = Q_2$.
			\item $\dot{Q}(t_1) = \dot{Q}_1$.
			\item $\dot{Q}(t_2) = \dot{Q}_2$.
		\end{itemize}
	\end{multicols}

	Now:

	\begin{multicols}{2}
		\begin{itemize}
			\item $\delta Q(t_1) = \delta Q(t_2) = 0$.
			\item $\delta\dot{Q}(t_1) = \delta\dot{Q}(t_2) = 0$.
		\end{itemize}
	\end{multicols}

	\begin{align*}
		\delta A &= \int_{t_1}^{t_2}\mathcal{L}(Q(t) + \delta Q(t), \dot{Q}(t)+\delta\dot{Q}(t))dt - \int_{t_1}^{t_2}\mathcal{L}(Q(t), \dot(Q)(t))dt = \\
						 &=\int_{t_1}^{t_2}\sum\limits_{\alpha=1}^{3N}\biggl[\frac{\partial\mathcal{L}}{\partial q_\alpha}\delta q_\alpha(t) + \frac{\partial\mathcal{L}}{\partial\dot{q}_\alpha}\delta\dot{q}_\alpha(t)\biggr]dt=\\
						 &=\sum\limits_{\alpha=1}^{3N}\frac{\partial\mathcal{L}}{\partial\dot{q}_\alpha}\delta q_\alpha(t)|_{t_1}^{t_2} + \int_{t_1}^{t_2}\sum\limits_{\alpha=1}^{3N}\biggl[\frac{\partial\mathcal{L}}{\partial q_\alpha}\delta q_\alpha(t) - \frac{d}{dt}\biggl(\frac{\partial\mathcal{L}}{\partial\dot{q}_\alpha}\biggr)\delta q_\alpha(t)\biggr] dt = 0
	\end{align*}

	Thus:

	$$\frac{\partial\mathcal{L}}{\partial q_\alpha} - \frac{d}{dt}\biggl(\frac{\partial\mathcal{L}}{\partial\dot{q}_\alpha}\biggr) = 0\Rightarrow \frac{d}{dt}\biggl(\frac{\partial\mathcal{L}}{\partial\dot{q}_\alpha}\biggr)-\frac{\partial\mathcal{L}}{\partial q_\alpha} = 0$$

\section{Constraints}

\begin{itemize}
	\item Holonomic constraints: $\sigma_k(q_1, \dots, q_{3N}, t) = 0\qquad k = 1, \dots, N_C$.
	\item Nonholonomic constraints: $\zeta(q_1, \dots, q_{3N}, \dot{q}_1, \dots, \dot{q}_{3N}) = 0$.
\end{itemize}

For example consider:

$$\frac{1}{2}\sum\limits_{i}m_i\dot{\vec{r}}_i^2-C = 0$$

Considering the minimal set of coordinates in generalized coordinates $3N-N_C$.
This is an example of spherical coordinates at fixed radius.

	\subsection{Differential forms}

	$$\sum\limits_{\alpha = 1}^{3N} a_{k\alpha}dq_\alpha + a_{kt}dt = 0\qquad k = 1, \dots, N_C$$


		\subsubsection{Holomonic constraints}

		$$\sum\limits_{\alpha=1}^{3N}\frac{\partial\sigma_k}{\partial q_\alpha}dq_\alpha + \frac{\partial\sigma_k}{\partial t} dt = 0\qquad k = 1, \dots, N_N\qquad a_{k\alpha} = \frac{\partial\sigma_k}{\partial q_\alpha}\qquad a _{kt} = \frac{\partial\sigma_k}{\partial t}$$

		\subsubsection{Nonholonomic constraints}

		$$\frac{1}{2}\sum\limits_i m_i\dot{\vec{r}}_i^2 - C = 0\Rigtharrow\frac{1}{2}\sum\limits_i m_i\dot{\vec{r}}_i\frac{d\vec{r}_i}{dt} - C = 0\Rightarrow\frac{1}{2}\sum\limits_i m_i\dot{\vec{r}}_id\vec{r}_i-Cdt = 0$$

		$$a_{1i}=\frac{1}{2}m_i\dot{\vec{r}}_i\qquad a_{1t} = -C$$

		So the integrable form:

		$$\sum\limits_{\alpha=1}^{3N}a_{k\alpha}dq_\alpha=0$$

			\paragraph{Lagrange multipliers}

			$$\int_{t_1}^{t_2}\sum\limits_{\alpha=1}^{3N}\biggl[\frac{\partial\mathcal{L}}{\partial q_\alpha} - \frac{d}{dt}\biggl(\frac{\partial\mathcal{L}}{\partial\dot{q}_\alpha}\biggr) + \sum\limits_{k=1}^{N_N}\lambda_ka_{k\alpha}\biggr]\delta q_\alpha(t)dt = 0$$

			$$\frac{d}{dt}\biggl(\frac{\partial\mathcal{L}}{\partial\dot{q}_\alpha}\biggr) - \frac{\partial\mathcal{L}}{\partial q_\alpha} = \sum\limits_{k=1}^{N_C}\lambda_k a_{k\alpha}$$

			$$\sum\limits_{\alpha=1}^{3N}a_{k\alpha}\dot{q}_\alpha + a_{kt} = 0\qquad k = 1, \dots, N_C$$

			So there are $3N+N_C$ equations for $3N+N_c$ unknowns.

	\subsection{Hamiltonian formulation}
	Considering the time-independent holonomic constraints:

	$$\begin{cases}\dot{q}_\alpha = \frac{\partial\mathcal{H}}{\partial p_\alpha}\\\dot{q}_\alpha = -\frac{\partial\mathcal{H}}{\partial q_\alpha} - \sum\limits_{k=1}^{N_C}\lambda_ka_{k\alpha}\\\sum\limits_{\alpha=1}^{3N}a_{k\alpha}\frac{\partial\mathcal{H}}{\partial p_\alpha} = 0\end{cases}$$

	So that:

	\begin{align*}
		\frac{d\mathcal{H}}{dt} &= \sum\limits_\alpha\biggl[\frac{\partial\mathcal{H}}{\partial q_\alpha}\dot{q}_\alpha +\frac{\partial\mathcal{H}}{\partial p_\alpha}\biggr] = \\
														&=\sum\limits_\alpha\biggl[\frac{\partial\mahtcal{H}}{\partial q_\alpha}\frac{\partial\mathcal{H}}{\partial p_\alpha} + \frac{\partial\mathcal{H}}{\partial p_\alpha}\biggl(\frac{\partial\mathcal{H}}{\partial q_\alpha} + \sum\limits_k \lambda_ka_{k\alpha}\biggr)\biggr] = \\
														&=\sum\limits_k\lambda_k\sum\limits_\alpha\frac{\partial\mathcal{H}}{\partial p_\alpha}a_{k\alpha} = 0
	\end{align*}

	So no work is done on a system by the imposition of holonomic constraints.

	\subsection{Constraints in a simulation}

	$$m_i\ddot{r}_i = \vec{F}_i + \sum\limits_{k=1}^{N_C}\lambda_k\nabla_i\sigma_k$$

	$$\frac{d}{dt}\sigma_k(\vec{r}_1, \dots, \vec{r}_N) = 0\Rightarrow\dot{\sigma}_k = \sum\limits_{i=1}^N\nabla_i\sigma_k\cdot\dot{\vec{r}}_i = 0$$

	Including the constraints on the integration algorithm:

	$$\vec{r}_i(\Delta t) = \vec{r}_i(0) + \Delta t\vec{v}_i(0) + \frac{\Delta t^2}{2m_i}\vec{F}_i(0) + \frac{\Delta t^2}{2m_i}\sum\limits_{k}\lambda_k\nabla_i\sigma_k(0)$$

	TO obtain $\lambda_k$:

	$$\vec{r}_i' = \vec{r}_i(0) + \Delta t\vec{v}_i(0) + \frac{\Delta t^2}{2m_i}\vec{F}_i(0)$$

	$$\vec{r}_i(\Delta t) = \vec{r}'_i + \frac{1}{m_i}\sum\limits_k\tilde{\lambda}_k\nabla_i\sigma_k(0)$$

	So that:

	$$\tilde{\lambda}_k = \frac{\Delta t^2}{2}\lambda_k$$

	\subsection{Constraint condition}

	$$\sigma_l)\vec{r}_1(\Delta t), \dots, \vec{r}_N(\Delta t)) = 0\qquad l = 1, \dots, N_C$$

	$$\sigma_l\biggl(\vec{r}_1' + \frac{1}{m_1}\sum\limits_k\tilde{\lambda}_k\nabla_1\sigma_k(0), \dots, \vec{r}_N' + \frac{1}{m_N}\sum\limits_k\tilde{\lambda}_k\nabla_N\sigma_k(0)\biggr) = 0\qquad l = 1, \dots, N_C$$

	Considering SHAKE, an iterative solution from an initial guess $\tilde{\lambda}_k^{(1)}$:

	$$\vec{r}_i^{(1)} = \vec{r}_i' + \frac{1}{m_i}\sum\limits_k\tilde{\lambda}_k^{(1)}\nabla_1\sigma_k(0)$$

	The exact solution: $\tilde{\lambda}_k = \tilde{\lambda}_k^{(1)} + \delta\tilde{\lambda}_k^{(1)}$:

	$$\vec{r}_i(\Delta t) = \vec{r}_i^{(1)} + \frac{1}{m_i}\sum\limits_k\delta\tilde{\lambda}_k^{(1)}\nabla_1\sigma_k(0)$$

	$$\sigma_l\biggl(\vec{r}_1^{(1)} = \frac{1}{m_1}\sum\limits_k\delta\tilde{\lambda}_k\nabla_1\sigma_k(0), \dots, \vec{r}_N^{(1)} + \frac{1}{m_N}\sum\limits_{k}\delta\tilde{\lambda}_k\nabla_N\sigma_k(0)\biggr) = 0$$

	Considering the Taylor series:

	$$\sigma_l(\vec{r}_1^{(1)}, \dots, \vec{r}_N^{(1)}) + \sum\limits_{i=1}^N\sum\limits_{k=1}^{N_C}\frac{1}{m_i}\nabla_i\sigma_l(\vec{r}_1^{(1)}, \dots, \vec{r}_N^{(1)})\nabla_i\sigma_k(\vec{r}_1(0), \dots, \vec{r}_N(0))\delta\tilde{\lambda}_k\approx 0$$

\section{Possible algorithms}

$$\sigma_l(\vec{r}_1^{(1)}, \dots, \vec{r}_N^{(1)}) + \sum\limits_{i=1}^N\sum\limits_{k=1}^{N_C}\frac{1}{m_i}\nabla_i\sigma_l(\vec{r}_1^{(1)}, \dots, \vec{r}_N^{(1)})\nabla_i\sigma_k(\vec{r}_1(0), \dots, \vec{r}_N(0))\delta\tilde{\lambda}_k\approx 0$$

\begin{itemize}
	\item Direct inversion: Matrix-shake or M-SHAKE: hte procedure must be repeated because the equation above is a linear approximation.
	\item Quick trick: only diagonal element are considered without any matrix inversion.
	\item The same procedure can be employed for velocities like in RATTLE.
	\item LINCS or linear constraint solver is based on the same principle and implemented in GROMACS.
\end{itemize}
