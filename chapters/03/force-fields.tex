\chapter{Semi-empirical force fields}

\section{Introduction}
A protein system can be modelled using a force field.
Whenever a simulation is started the interactions between atoms need to be determined.
To do so a topology file is built.
This file will contain the connecting information.
This is important because the kind of force field is specified here.
Semi-empirical force field will be used because a very complicated process is modelled using formula that can be easily computed.
These formulae are not rigorous and what it is done is to compare the result of the simulation with an experiment.
Because of these the used force fields will be called semi-empirical.

\section{Potential energy surface}
Any system will correspond to a given potential energy surface.
There are states which correspond to minima of this energy where the system will spend most of its time.
In the global minimum the system is expected to be in equilibrium.
There are also local minima where the system can spend some time and the system will go from the global to some other local minima and the time spent there depends on how deep that minimum is.
There will be transitions from one minimum to another exploring the potential energy surface passing through a set of points.
In principle the potential energy surface is known because it is assumed that there is a potential energy for all the interaction that there are in the system.
This is not that simple because of temperature: at $0$ temperature the potential energy surface will tell everything: the system goes through the global minimum and stay there.
At finite and high temperatures the system will jump with a certain frequency, which depends on temperature, from one state to another.
In this case statistical mechanics is necessary not the potential energy surface is necessary but also the free energy surface and the entropy have to be taken into account.
The entropy part is the most difficult.
In order to compute entropy the conformational space need to be explored: so the compatible conformation in certain condition need to be found.
The phase space, the possibilities of a molecule need to be explored.
These are so many in the case of a protein that estimating entropy will be the more costly process.
In order to estimate entropy long simulation are needed as the conformational space will be explored as much as possible.
In this chapter entropy will not be considered.

The potential energy surface or PES is the landscape of what values the potential energy of an atom can assume.
Different points can be recognized like:

\begin{multicols}{2}
	\begin{itemize}
		\item Saddle point.
		\item Local maximums.
		\item Local minimum.
	\end{itemize}
\end{multicols}

	\subsection{Bond stretching}
	In figure \ref{chem-bond} the typical potential energy for a chemical bond can be seen.
	Let $r_{eq}$ the distance for which the energy is minimal, where the equilibrium is.
	Moving from it the energy increases.
	There are transitions between different states and in that case quantum physics should be used.
	Everything will be assumed to be described using classical mechanics.
	This is valid when dealing with conformational transitions.
	SO all variables are continuous.
	Using this assumption light absorption and a chemical reactions cannot be described.
	The potential energy surface to describe an unbreakable bond can be seen in figure \ref{chem-bond}.
	When close to the minimum the well can be assume symmetric, but some asymmetry later: the left part is steeper.
	An approximation for these potential energy is built into the force field.
	The potential energy surface can be reconstructed using vibrational spectroscopy and then it is reconstructed using a mathematical model.
	Using a Taylor expansion the potential energy is approximated at point $r$ by taking the value at the equilibrium and then constructing all the corrections.
	The first one is the first derivative at the value at equilibrium multiplied by the distance.
	Then for the second correction the second derivative and so on.
	If the potential is symmetric the first derivative is $0$, as it happens near the equilibrium.
	So the first correction can be neglected.
	Also the third order term is either $0$ because the third derivative is $0$ if the minimum is really shallow.
	This term is really small if $r$ is really close to the equilibrium so it can neglected.

	\begin{align*}
		U(r)&=U(r_{eq}) + \frac{dU}{dr}|_{r=r_{eq}}(r-r_{eq})+\frac{1}{2!}\frac{d^2U}{dr^2}|_{r=r_{eq}}(r-r_{eq}^2)+\frac{1}{3!}\frac{d^3U}{dr^3}|_{r=r_{eq}}(r-r_{eq}^3)+\cdots\\
		U(r)&=U(r_{eq}) + \xcancel{\frac{dU}{dr}|_{r=r_{eq}}(r-r_{eq}})+\frac{1}{2!}\frac{d^2U}{dr^2}|_{r=r_{eq}}(r-r_{eq}^2)+\xcancel{\frac{1}{3!}\frac{d^3U}{dr^3}|_{r=r_{eq}}(r-r_{eq}^3)}+\cdots\\
	\end{align*}

	So that in the end the typical harmonic potential is obtained.

	$$U(r_{AB}) = \frac{1}{2}k_{AB}(r_{AB}-r_{AB,eq})^2$$

	The constant $k$ is related to the second derivative with respect to the distance.
	The distances and the $k$ constants are labelled with $A$ and $B$, which stand for the fact that this interaction has to be described for each couple of atom.
	Transferability of these parameter is an issue: using a particular force field then the parameters cannot be used for another.
	So each force field will come with its own set of parameters.

		\subsubsection{Anharmonic force constant}
		Considering that the shape is not completely symmetric the third order term can be inserted because it can be important.
		This introduces an asymmetry in the system: an anharmonic force constant.

		$$U(r_{AB}) = \frac{1}{2}[k_{AB}+k^{(3)}_{AB}(r_{AB}-r_{AB, eq})](r_{AB}-r_{AB, eq})^2$$

		\subsubsection{Quartic correction}
		Also the fourth order term can be inserted.

		$$U(r_{AB}) = \frac{1}{2}[k_{AB}+k^{(3)}_{AB}(r_{AB}-r_{AB, eq}) + k^{(4)}_{AB}(r_{AB}-r_{AB,eq})^2](r_{AB}-r_{AB, eq})^2$$

		\subsubsection{Morse potential}
		The Morse potential is used in implicit solvent simulation and uses the exponential because it can describe screen interaction that happens with implicit solvent.
		The exponential is quite expensive for a computer to compute.
		It is useful also for soft-system or coarse grained system.

		$$U(r_{AB}) = D_{AB}[1-e^{-\alpha_{AB}(r_AB-r_{AB,eq})^2}]$$

\section{Valence angle bending}
Chemical bonds are not described only by strings and beads but also bending of the bonds can happen.
These are valence angle bending.
In order to describe them a Taylor expansion introducing and equilibrium angle is built.
The first order term is not considered as it will be equal to $0$.
Then the second introduces the harmonic and the third for the anharmonic one.
This formula is similar to the previous one but all the constant have to be described between each triplet of atoms.

$$U(\theta_{ABC}) = \frac{1}{2}[k_{ABC}+k^{(3)}_{ABC}(\theta_{ABC}-\theta_{ABC,eq})+k^{(4)}_{ABC}(\theta_{ABC}-\theta_{ABC,eq})^2+\cdots](\theta_{ABC}-\theta_{ABC, eq})^2$$

\section{Torsions}

$$U(\omega_{ABCD}) = \frac{1}{2}\sum\limits_{\{j\}_{ABCD}}V_{j,ABCD}[1+(-1)^{j+1}\cos(j\omega_{ABCD}+\psi_{j,ABCD})]$$

	\subsection{Improper torsions}

$$U(\omega_{ABCD}) = \frac{1}{2}\sum\limits_{\{j\}_{ABCD}}V_{j,ABCD}[1+(-1)^{j+1}\cos(j\omega_{ABCD}+\psi_{j,ABCD})]$$

\section{Van der Waals interactions}
Some force fields reduce $1,4$-interactions by a scale factor through torsions.

	\subsection{Lennard-Jones potential}

	\begin{align*}
		U_(r_{AB}) &= \frac{a_{AB}}{r^{12}_{AB}}-\frac{b_{AB}}{r^6_{AB}}=\\
							 &= 4\epsilon_{AB}\biggl[\biggl(\frac{\sigma_{AB}}{r_{AB}}\biggr)^{12}-\biggl(\frac{\sigma_{AB}}{r_{AB}}\biggr)^6\biggr]
	\end{align*}

	And the distance with minimum energy:

	$$r^*_{AB} = 2^{\frac{1}{6}}\sigma_{AB}$$

	\subsection{Morse potential}

	$$U(r_{AB}) = D_{AB}[1-e^{-a_{AB}(r_{AB}-r_{AB,eq})^2}]$$

	\subsection{Hill potential}

	$$U(r_{AB}) = \epsilon_{AB}\biggl[\frac{6}{\beta_{AB}-6}e^{\beta_{AB}\frac{1-r_{AB}}{r^*_{AB}}}-\frac{\beta_{AB}}{\beta_{AB}-6}\biggl(\frac{r^*_{AB}}{r_{AB}}\biggr)^6\biggr]$$


\section{Electrostatic interactions}
Consider all the electrostatic interactions:

$$U_{AB} = \vec{M}^{(A)}V^{(B)}$$

Now, summing over all molecules:

$$U_{AB} = \sum\limits_{A}\sum\limits_{B>A}\vec{M}^{(A)}\vec{V}^{(B)}$$

	\subsection{Point like charges}
	All atoms are considered as point-like charges:

	$$U_{AB} = \frac{q_Aq_B}{\epsilon_{AB}r_{AB}}$$

	\subsection{Dipolar interactions}

	$$U_{AB/CD} = \frac{\mu_{AB}\mu_{CD}}{\epsilon_{AB/CD}r^3_{AB/CD}}(\cos\chi_{AB/CD}-3\cos\alpha_{AB}\cos\alpha_{CD})$$

	\subsection{Dielectric constants}

	$$U_{AB} = \frac{q_Aq_B}{\epsilon_{AB}r_{AB}}$$

	$$\epsilon_{AB} = \begin{cases}\infty&\text{ if }A\land B\text{are 1,2- or 1,3-related}\\3.0&\text{ if }A\land B\text{are 1,4-related}\\1.5&\text{otherwise}\end{cases}$$

	\subsection{Cross terms}

	\begin{align*}
		U(\vec{q}) = &U(\vec{q}_{eq}) + \sum\limits_{i=1}^{3N-6}(q_i-q_{i,eq})\frac{\partial U}{\partial q_i}|_{\vec{q}=\vec{q}_{eq}} + \\
								 &+\frac{1}{2!}\sum\limits_{i=1}^{3N-6}\sum\limits_{j=1}^{3N-6}(q_i-q_{i.eq})(q_j-q_{j,eq})\frac{\partial^2 U}{\partial q_i\partial q_j}|_{\vec{q}=\vec{q}_{eq}} +\\
								 &=\frac{1}{3!}\sum\limits_{i=1}^{3N-6}\sum\limits_{j=1}^{3N-6}\sum\limits_{k=1}^{3N-6}(q_i-q_{i,eq})(q_j-q_{j.eq})(q_k-q_{k,eq})\frac{\partial^3 U}{\partial q_i\partial q_j\partial q_k}|_{\vec{q}=\vec{q}_{eq}} + \cdots
	\end{align*}

	$$U(r_{AB}, \theta_{ABC}) = \frac{1}{2}k_{AB,ACB}(r_{AB}-r_{AB, eq})(\theta_{ABC}-\theta_{ABC, eq})$$

	\subsection{Parametrization}
	Let:

	$$Z = \biggl[\sum\limits_{i}^{observables}\sum\limits_{j}^{occurrences}\frac{(calc_{i,j}-expt_{i,j})^2}{w_i^2}\biggr]^{\frac{1}{2}}$$

	The penalty function.
	Consider the number of parameters:

	$$p = N + (N-1)+(N-2)+\cdots = N\frac{N+1}{2}$$

	A possible strategy for parametrization is:

	\begin{align*}
		\sigma_{AB} &= \sigma_A+\sigma_B
		\epsilon_{AB} &= (\epsilon_A\epsilon_B)^{\frac{1}{2}}
	\end{align*}

\section{Force field energies}

	\subsection{Geometry optimization}

	$$\vec{g}(\vec{q}) = \begin{bmatrix} \frac{\partial U}{\partial q_1} \\ \frac{\partial U}{\partial q_2} \\ \vdots \\ \frac{\partial U}{\partial q_n}\end{bmatrix}$$

	Such that the cost reaches a global minimum $J_{min}(\vec{w})$.

	\subsection{Derivative of the potential function}

	$$\frac{\partial U}{\partial x_A} = \sum\limits_{i\in A}\frac{\partial U}{\partial r_{Ai}}\frac{r_{Ai}}{\partial x_A}$$

	$$U(r_{AB}) = \frac{1}{2}[k_{AB}+k_{AB}^{(3)}(r_{AB}-r_{AB, eq}) + k_{AB}^{(4)}(r_{AB}-r_{AB, eq})^2](r_{AB}-r_{AB,eq})^2$$

	$$\frac{\partial U}{\partial r_{Ai}} = \frac{1}{2}[2k_{Ai}+3k_{Ai}^{(3)}r_{Ai}-r_{Ai, eq}) + 4k^{(4)}_{Ai}(r_{Ai}-r_{Ai, eq})^2](r_{Ai}-r_{Ai, eq})$$

	$$\frac{\partial r_{Ai}}{\partial x_A} = \frac{x_A-x_i}{\sqrt{(x_A-x_i)^2+(y_A-y_i)^2+(z_A-z_i)^2}}$$

		\subsubsection{Newton-Raphson}

		$$U(\vec{q}^{(k+1)}) = U(\vec{q}^{(k)}) + (\vec{q}^{(k+1)} -\vec{q}^{(k)})\vec{g}^{(k)} + \frac{1}{2}(\vec{q}^{(k+1)}-\vec{q}^{(k)})H^{(k)}(\vec{q}^{(k+1)}-\vec{q}^{(k)})$$

		Where $H$ is the Hessian matrix built:

		$$H_{ij}^{(k)} = \frac{\partial^2 U}{\partial q_i\partial q_j}|_{\vec{q}=\vec{q}^{(k)}}$$

		And:

		$$\frac{\partial U(\vec{q}^{(k+1)})}{\partial q_i^{(k+1)}} = \frac{\partial \vec{q}^{(k+1)}}{\partial q_i^{(k+1)}}\vec{g}^{(k)}+ \frac{1}{2}\frac{\partial \vec{q}^{(k+1)}}{\partial q_i^{(k+1)}}H^{(k)}(\vec{q}^{(k+1)}-\vec{q}^{(k)})+\frac{1}{2}(\vec{q}^{(k+1)}-\vec{q}^{(k)})H^{(k)}\frac{\partial \vec{q}^{(k+1)}}{\partial q_i^{(k+1)}}$$

		Where:

		$$\vec{g}_i^{(k+1)} = \vec{g}_i^{(k)} + [H^{(k)}(\vec{q}^{(k+1)}-\vec{q}^{(k)})]_i$$

		With the stationary condition:

		$$\vec{0} = \vec{g}^{(k)} + H^{(k)}(\vec{q}^{(k+1)}-\vec{q}^{(k)})\Rightarrow \vec{q}^{(k+1)} = \vec{q}^{(k)} - [H^{(k)}]^{-1}\vec{g}^{(k)}$$

	\subsection{Types of force fields}
	Force fields can be categorized as:

	\begin{multicols}{2}
		\begin{itemize}
			\item All atoms: one atom corresponds to one bead.
			\item More atoms: more atoms correspond to one bead.
			\item Corse grained: groups of atoms correspond to one bead.
			\item Polarizable force fields: point charges are variables.
		\end{itemize}
	\end{multicols}

	As a golden rule parameters from different force fields should never be mixed.
