\chapter{Thermostats}

\section{Introduction}

	\subsection{Velocity rescaling}

	Target kinetic energy, where $N_f$ is the number of degrees of freedom.

	$$\bar{K} = \frac{N_f}{2\beta}$$

	The actual kinetic energy is:

	$$K = \frac{1}{2}\sum\limits_i m_i\vec{v}_i^2$$

	Considering a rescaling factor $\alpha = \sqrt{\frac{\bar{K}}{K}}$ where $\vec{v}_i \rightarrow \frac{\vec{v}_i}{\alpha}$ at a predetermined frequency.
	The rescaled kinetic energy:

	$$\frac{1}{2}\sum\limits_{i}m_i\frac{\vec{v}_i^2}{\alpha^2} = \frac{\bar{K}}{2K}\sum\limits_{i}m_i\vec{v}_i^2 = \bar{K}$$

	This is rigorously correct in the thermodynamics limit: $N\rightarrow\infty$

\section{Type of thermostats}

	\subsection{Andersen's thermostat}
	Consider a collision frequency $\nu$, then $\nu\Delta t$ is the probability of a collision in $\Delta t$.
	Picking a random number $\rho$ from a uniform probability distribution in $[0,1]$, if $\rho<\nu\Delta t$ the velocity is reassigned to the particle, extracting it from the Maxwell-Boltzmann distribution:

	$$P(p) = \biggl(\frac{\beta}{2\pi m}\biggr)^{\frac{3}{2}}e^{-\beta\frac{p^2}{2m}}$$

	\subsection{Andersen revisited}
	The target kinetic every $K_t$ is selected from a distribution:

	$$P(K_t)dK_t\propto K_t^{\frac{N_f}{2}-1}e^{-\beta K_t}d K_t$$

	Considering a rescaling factor $\alpha = \sqrt{\frac{K_t}{K}}$ where $\vec{v}_i \rightarrow \frac{\vec{v}_i}{\alpha}$ at a predetermined frequency.
	The rescaled kinetic energy:

	$$\frac{1}{2}\sum\limits_{i}m_i\frac{\vec{v}_i^2}{\alpha^2} = \frac{K_t}{2K}\sum\limits_{i}m_i\vec{v}_i^2 = K_t$$

	This is rigorously correct in the thermodynamics limit: $N\rightarrow\infty$

	\subsection{Langevin thermostat}
	Consider a friction coefficient $\gamma$ and random velocities $\eta_i$.
	The properties of the random velocities are:

	\begin{multicols}{2}
		\begin{enumerate}
			\item Collisions between solute and thermal bath particles are many and independent.
				Components of the random velocities are sampled from a Gaussian distribution.
			\item On average there is no net momentum exchange between the thermal bath and the solute particles: $\langle\eta_i(t)\rangle = 0$.
			\item Frequent and instantaneous collisions: the auto correlation function of the velocities decays fast: $\langle\eta_i(t)\eta_j(t)\rangle = 6D\delta_{ij}\delta(t-t')$ where $D = \frac{kT}{m\gamma}$ is the diffusion coefficient.
		\end{enumerate}
	\end{multicols}

	The equations of motion are coupled overdamped Langevin equations:

	$$\dot{\vec{r}}_i(t) = \beta D\vec{F}_i(\vec{r}_1(t), \dots, \vec{r}_N(t)) + \eta_i(t)$$

	The configurations sample a probability distribution that is the solution of the Fokker-Planck equation:

	$$\frac{\partial}{\partial t}P(\vec{x}, t) = D\sum\limits_{i=1}^{N}\nabla_i\cdot[\nabla_i+\beta\nabla\mathcal{H}(\vec{x})]$$

	The Boltzmann distribution $P(\vec{x}) = Ce^{-\beta\mathcal{H}(\vec{x})}$ is the stationary solution of the Fokker-Planck equation.
	To integrate the Langevin equation:

	$$\dot{\vec{r}}_i(t) = \beta D\vec{F}_i(\vec{r}_1(t), \dots, \vec{r}_N(t)) + \eta_i(t)$$

	$$\langle\eta_i(t)\eta_j(t)\rangle = 6D\delta_{ij}\frac{\delta_{kl}}{\Delta t}$$

	The random velocity is sampled from a Gaussian distribution with a non-unitary variance:

	$$\sigma^2 = \langle\eta_i(0)\eta_i(0)\rangle = \frac{6D}{\Delta t}$$

	A random force $\xi_i(t_k)$ satisfying the unitary variance condition:

	$$\xi_i(t_k) = \sqrt{\frac{\Delta t}{6D}}\eta_i(t_k)\qquad\langle\xi_i(t_k)\rangle = 0\qquad\langle\xi_i(t_k)\xi_i(t_l)\rangle = \delta_{ij}\delta_{kl}$$

	Considering a discretized Langevin equation:

	$$\frac{\vec{r}_i(t+ \Delta t) - \vec{r}_i(t)}{\Delta t} = \beta D\vec{F}_i(\vec{r}_1(t), \dots, \vec{r}_N(t)) + \sqrt{\frac{6D}{\Delta t}}\xi_i(t)$$

	Thus:

	$$\vec{r}_i(t + \Delta t) = \vec{r}_i(t) + \Delta t\frac{\vec{F}_i(\vec{r}_1(t), \dots, \vec{r}_N(t))}{m_i\gamma} + \sqrt{\frac{6kT\Delta t}{m_i\gamma}}\xi_i(t)$$

\section{Bussi velocity Verlet}
The Bussi velocity Verlet is based on velocity rescale with the following algorithm:

\begin{multicols}{2}
	\begin{itemize}
		\item Evolve the system for a single time step with Hamilton's equations, using a time-reversible area-preserving integration scheme like the velocity Verlet.
		\item Compute the kinetic energy.
		\item Evolve the kinetic energy for a time corresponding to a single time step using an auxiliary continuous stochastic dynamics.
		\item Rescale the velocities so as to enforce this new value of the kinetic energy.
	\end{itemize}
\end{multicols}

Auxiliary dynamics:

$$dK = \biggl(D(K)\frac{\partial\log P(K)}{\partial K}+\frac{\partial D(K)}{\partial K}\biggr)dt + \sqrt{2D(K)}dW$$

Inserting the distribution from Heyes' algorithm: $P(K_t)dK_t\propto K_t^{\frac{N_f}{2}-1}e^{\beta K_t}dK_t$:

$$dK = \biggl(\frac{N_f D(K)}{2K\bar{K}}(K-\bar{K}) - \frac{D(K)}{K} + \frac{\partial D(K)}{\partial K}\biggr) dt + \sqrt{2D(K)}dW$$

A possible choice is $D(K) = \frac{2K\bar{K}}{N_f\tau}$:

$$dK = (K-\bar{K})\frac{dt}{\tau}+\sqrt{\frac{2K\bar{K}}{N_f}}\frac{dW}{\sqrt{\tau}}$$

Neglecting the stochastic term Berendsen's thermostat is obtained:

$$dK = (K-\bar{K})\frac{dt}{\tau}$$

$$\tau\rightarrow 0\Rightarrow \tau dK = (K-\bar{K})dt + \sqrt{\frac{2K\bar{K}\tau}{N_f}}dW\Rightarrow (K-\bar{K})dt = 0 \text{ (Heyes)}$$

$$\tau\rightarrow\infty\Rightarrow dK = (K-\bar{K})\frac{dt}{\tau} + \sqrt{\frac{2K\bar{K}}{N_f}}\frac{dW}{\sqrt{\tau}}\Rightarrow dK = 0\text{ (Hamiltonian dynamics)}$$

Solution to the differential equation leads to a scaling factor $R_i$ independent Gaussian random numbers:

$$\alpha^2 = e^{-\frac{\Delta t}{\tau}} + \frac{\bar{K}}{N_fK}(1-e^{-\frac{\Delta t}{\tau}})\sum\limits_{i=1}^{N_f}R_i^2+2e^{-\frac{\Delta t}{2\tau}}\sqrt{\frac{\bar{K}}{N_f K}(1-e^{-\frac{\Delta t}{\tau}}R_1)}$$

\section{Nos\`e Hamiltonian}
Extended phase space:

$$\mathcal{H}_N = \sum\limits_i\frac{\vec{p}_i^2}{2m_is^2} + U(\vec{r}_1, \dots, \vec{r}_n)_ \frac{p_s^2}{2Q}+ gkT\log s$$

Microcanonical partition function:

$$\Omega = \int d^N\vec{r}d^N\vec{p}dsdp_s\delta\biggl(\sum\limits_{i}\frac{\vec{p}_i^2}{2m_is^2} + U(\vec{r}_1, \dots, \vec{r}_N) + \frac{p_s^2}{2Q} + gkT\log s - E\biggr)$$

Scaling the momentum $\frac{\vec{p}}{s}\rightarrow \vec{p}$:

$$\Omega = \int d^N\vec{r}d^N\vec{p}dsdp_ss^{dN}\delta\biggl(\mathcal{H}+\frac{p_s^2}{2Q} + gkT\log s - E\biggr)$$

This can be solved using the property of the $\delta$ function.

$$\delta(f(s)) = \frac{\delta(s-s_0)}{|f'(s_0)|}\qquad f(s) = \mathcal{H}+\frac{p_s^2}{2Q} + gkT\log s - E\qquad f'(s) = \frac{gkT}{s}$$

$$s_0 = e^{\frac{1}{gkT}\bigl(E-\mathcal{H}-\frac{p_s^2}{2Q}\bigr)}\Rightarrow f'(s_0) = gkTe^{-\frac{1}{gkT}\bigl(E-\mathcal{H}-\frac{p_s^2}{2Q}\bigr)}$$

$$\delta(f(s)) = \frac{1}{gkT}e^{\frac{1}{gkT}\bigl(E-\mathcal{H}-\frac{p_s^2}{2Q}\bigr)}\delta\biggl(s-e^{\frac{1}{gkT}\bigl(E-\mathcal{H}-\frac{p_s^2}{2Q}\bigr)}\biggr)$$

$$\Omega = \int d^N\vec{r}d^N\vec{p}dsdp_ss^{dN}\delta\biggl(\mathcal{H}+\frac{p_2^2}{2Q}+gkT\log a-E\biggr)$$

$$\delta(f(s)) = \frac{1}{gkT}e^{\frac{1}{gkT}\bigl(E-\mathcal{H}-\frac{p_s^2}{2Q}\bigr)}\delta\biggl(s-e^{\frac{1}{gkT}\bigl(E-\mathcal{H}-\frac{p_s^2}{2Q}\bigr)}\biggr)$$

$$\Omega = \frac{1}{dkT}\int d^N\vec{r}d^N\vec{p}dp_se^{\frac{dN+1}{gkT}\bigl(E-\mathcal{H}-\frac{p_s^2}{2Q}\bigr)}$$

$$g = dN + 1\Rightarrow\Omega = \frac{e^{\frac{E}{kT}}\sqrt{2\pi QkT}}{(dN+1)kT}\int d^N\vec{r}d^N\vec{p}e^{-\frac{\mathcal{H}}{kT}}$$

	\subsection{Nos\`e equations}

	$$\mathcal{H}_N = \sum\limits_i\frac{\vec{p}_i^2}{2m_is^2}+U(\vec{r}_1,\dots, \vec{r}_N) + \frac{p_s^2}{2Q} + gkT\log s$$

	$$\dot{\vec{r}}_i = \frac{\partial\mathcal{H}_N}{\partial\vec{p}_i} = \frac{\vec{p}_i}m_is^2\qquad \dot{\vec{p}}_i = -\frac{\partial\mathcal{H}_n}{\partial\vec{r}_i} = -\frac{\partial U}{\partial\vec{r}_i} = \vec{F}_i$$

	$$\dot{s} = \frac{\partial\mathcal{H}_N}{\partial p_s} = \frac{p_s}{Q}\qquad \dot{p}_s = -\frac{\partial\mathcal{H}_n}{\partial S} = \sum\limits_i\frac{\vec{p}_i^2}{m_is^3} - \frac{gkT}{s} = \frac{1}{s}\biggl[\sum\limits_i\frac{\vec{p}_i^2}{m_is^2}-gkT\biggr]$$

	\subsection{Nos\`e-Hoover equations}
	Noncanonical change of variables:

	$$\vec{p}_i' = \frac{\vec{p}_i}{s}\qquad p_s' = \frac{p_s}{s}\qquad dt' = \frac{dt}{s}$$

	Remembering:

	\begin{multicols}{2}
		\begin{itemize}
			\item $\dot{\vec{r}}_i = \frac{\vec{p}_i}{m_i}$.
			\item $\dot{\vec{p}}_i = \vec{F}_i-\frac{p_\eta}{Q}\vec{p}_i$.
			\item $\dot{\eta} = \frac{p_\eta}{Q}$.
			\item $\dot{p}_\eta = \sum\limits_{i}\frac{\vec{p}_i^2}{m_i}-dNkT$.
		\end{itemize}
	\end{multicols}

	Now:

	$$\frac{d\vec{r}_i}{dt} = \frac{\vec{p}_i}{m_is^2}\Rightarrow\frac{d\vec{r}_i}{dt'} = \frac{\vec{p}_i'}{m_i}\qquad \frac{d\vec{p}_i'}{dt'} = \vec{F}_i-\frac{sp'_s}{Q}\vec{p}_i'$$

	$$\frac{ds}{dt'} = \frac{s^2p_s'}{Q}\qquad \frac{1}{s}\frac{ds}{dt'} = \frac{d\eta}{dt'}$$

	$$p_s = p_\eta = sp'_s\qquad \frac{dp_s'}{dt'} = \frac{1}{s}\biggl[\sum\limits_{i} \frac{(\vec{p}_i')^2}{m_i}-gkT\biggr] - \frac{s(p_s')^2}{Q}$$
